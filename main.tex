\documentclass{article}

% Packages
\usepackage[utf8]{inputenc}
\usepackage{hyperref}
\usepackage{amsthm}
\usepackage{amssymb}
\usepackage{amsmath}
\usepackage{algorithm}
\usepackage{algpseudocode}
\usepackage{bm}
\usepackage{xargs}                      % Use more than one optional parameter in a new commands
\usepackage[pdftex,dvipsnames]{xcolor}  % Coloured text etc.
\usepackage[colorinlistoftodos,prependcaption,textsize=tiny]{todonotes}

% Definitions, Theorems, etc.
\theoremstyle{definition}
\newtheorem{definition}{Definition}[section]

\newtheorem{example}{Example}[section]
\newtheorem{theorem}{Theorem}[section]
\newtheorem{lemma}[theorem]{Lemma}

% Metadata
\title{Discrete Mathematics VO Script}
\author{Daniel Schloms\\
        TU Wien}
\date{}

% TODO command
\newcommandx{\missing}[2][1=]{\todo[linecolor=Plum,backgroundcolor=Plum!25,bordercolor=Plum,#1]{#2}}

\begin{document}

\maketitle

\section{Introduction}

This script compiles the contents of the VO Discrete Mathematics into a document.

\section{Graph Theory}

In this lecture only finite graphs are covered, therefore any graph here can be considered finite.

\subsection{Basics}

\begin{definition}[Simple Graph]
A Simple Graph $\bm{G = (V, E)}$ consists of \textbf{Vertices} $\bm{V}$ and \textbf{Edges} $\bm{E \subseteq \{\{u, v\} \vert u, v \in V\}}$.
\end{definition}

\begin{definition}[Adjacency]
Vertices $u, v$ of an Edge $\{u, v\}$ are \textbf{adjacent}.
\end{definition}

\begin{definition}[Incidence]
Vertex $v$ is \textbf{incident} to edges $\{u, v\}$ containing that vertex $v$. 
\end{definition}
\
\missing[inline]{Loop definition}\
\missing[inline]{Multigraph definition}\
\missing[inline]{Weighted Graph definition}\
\missing[inline]{Neighbor definition}\

\begin{definition}[Degree of a Vertex]
Degree $d(u) = \{e \in E \vert u \text{ incident to } e\}$
\end{definition}

\begin{definition}[Regularity of a Graph]
A graph is regular of degree $r$ if $d(u) = r$ for all $u \in V$.
\end{definition}

\begin{definition}[Directed (Multi-)Graph]
A \textbf{directed (multi-)graph} is a graph where every edge has a \textbf{head} and a \textbf{tail}. Or alternatively, edges are \textbf{pairs} $\bm{(u, v)}$ of vertices, not sets. In a directed graph there is an additional distinction between \textbf{incoming degree} $\bm{d^+(u)}$ and \textbf{outgoing degree} $\bm{d^-(u)}$.
\begin{gather*}
d^+(u) = \vert \{e \in E \vert e = (u, v) \text{ for some } v \in V\} \\
d^-(u) = \vert \{e \in E \vert e = (v, u) \text{ for some } v \in V\}
\end{gather*}
\end{definition}

\begin{lemma}[Handshaking Lemma] 
\[\displaystyle 
\sum_{v \in V}^{} d(v) = 2 \vert E \vert
\]
\end{lemma}

\begin{definition}[Degree Matrix]
The \textbf{degree matrix} $D$ is a matrix, where the diagonal elements $a_{ii}$ are the degree of the vertex $i$, the other elements are 0.
\end{definition}

\begin{definition}[Laplacian Matrix]

Laplacian matrix $L = D - A$, where $D$ is the \textbf{degree matrix}, and $A$ is the \textbf{adjacency matrix}.

\end{definition}

\begin{theorem}[Kirchhoff's Matrix Tree Theorem]
For a given connected graph $G$ with $n$ labeled vertices, let $\lambda_1, \lambda_2, \ldots, \lambda_{n-1}$ be the non-zero eigenvalues of its Laplacian matrix. Then the number of spanning trees of $G$ is 
\[
t(G)=\frac{1}{n} \lambda_1\lambda_2\cdots\lambda_{n-1}.
\]

\end{theorem}

\subsection{Spanning Tree Algorithms}

\begin{algorithm}[H]
\caption{Kruskal's Algorithm}
\begin{algorithmic}
\State $G = (V, E) \gets$ The original graph
\State $S \gets \varnothing$ \Comment{Empty set of edges}
\While{$(V, S)$ not a spanning tree}
    \State $e \gets$ edge from $E$ with minimal weight not in S
    \State Remove $e$ from $E$
    \If{$(V, S \cup \{e\}$) loop-free}
        \State Add $e$ to $S$
    \Else
        \State Discard $e$
    \EndIf
\EndWhile
\end{algorithmic}
\end{algorithm}

\begin{algorithm}[H]
\caption{Prim's Algorithm}
\begin{algorithmic}
\State $G = (V, E) \gets$ The original graph
\State $S \gets \varnothing$ \Comment{Empty set of edges}
\State $N \gets \{v_0\}$ \Comment{Set with a starting vertex}
\While{$N \neq V$}
    \State $e \gets$ edge from $E$ with min. weight $\notin$ S and adjacent to any $v \in N$
    \State Remove $e$ from $E$
    \If{For $e = (v, u)$, $u$ not in $N$}
        \State Add $e$ to $S$
        \State Add $u$ to $N$
    \Else
        \State Discard $e$
    \EndIf
\EndWhile
\end{algorithmic}
\end{algorithm}

\subsection{Shortest Path Algorithms}

\missing[inline]{Dijkstra}\
\missing[inline]{Bellman-Ford}\
\missing[inline]{Floyd-Warshall}

\subsection{Matroids}

\begin{definition}[Matroid]\label{def:matroid}
    A matroid is a pair $(E, I)$ where $E$ is a finite \textbf{ground set} and $I$ is a family of subsets (\textbf{independent sets}) of $E$. The following properties hold (independence axioms):
    \begin{itemize}
        \item[1)] $\varnothing \in I$
        \item[2)] $B \in I$, $A \subseteq B \implies A \in I$ (downward-closedness)
        \item[3)] $I_1, I_2 \in I$, $|I_1| < |I_2| \implies \exists e \in I_2 - I_1$, $I_1 \cup \{e\} \in I$ (the difference of $I_1$ and $I_2$ contains an element that, when combined with the smaller set, is in $I$)
    \end{itemize}
\end{definition}

\begin{theorem}[Cardinality of Bases of a Matroid]
    All bases of a matroid have the same cardinality. A base can be defined as a maximal set in $I$, therefore this must be true due to the third independence axiom seen in Definition \ref{def:matroid}
\end{theorem}

\subsection{Flow Networks}

\missing[inline]{Flow}\
\missing[inline]{Cut}\

\begin{definition}[Capacity of a Cut]
    Let $S$, $T$ be subsets of $V$ of a graph $(V, E)$, i.e. a cut with $s \in S$ and $t \in T$.
    Then the \textbf{capacity} of this cut $C(S, T)$ is the sum of weights of edges going crossing from $S$ to $T$.
    \[
    C(S, T) = \sum_{\substack{  (u, v) \in E \\ u \in S \\ v \in T}} w(u, v)
    \]
\end{definition}

\begin{theorem}[Value of Flow]
    
\end{theorem}

\begin{theorem}[Min-Flow-Max-Cut]
    In a flow network, the maximum amount of flow passing from source $s$ to sink $t$ is equal to the total weight of the edges in a minimum cut, i.e. the smallest total weight of edges needed to disconnect $s$ from $t$.
\end{theorem}

\subsubsection{Flow Algorithms}

\missing[inline]{Ford-Fulkerson}\
\missing[inline]{Edmonds-Karp}

\subsection{Graph Coloring}

\begin{definition}[Chromatic Number]
    The \textbf{chromatic number} $\mathcal{X}(G)$ is the min. number of colors in a proper vertex coloring of $G$.
\end{definition}

\begin{theorem}[4-Color Theorem]
    \[\displaystyle 
    G \text{ is planar} \iff \mathcal{X}(G) \leq 4
    \]
\end{theorem}

\input{combinatorics.tex}

\section{Generating Functions}

\begin{example}[Generating Functions for Recurrence Relation]

\end{example}

\section{Partially Ordered Sets}

\begin{definition}[Strict/Irreflexive Partial Order]
    A strict partial order is a set $P$ with a relation $<$ such that the following properties hold.
    \begin{itemize}
        \item Asymmetry: $a < b \implies b \nless a$
        \item Transitivity: $a < b \land b < c \implies a < c$  
        \item Irreflexivity: $a \nless a$
    \end{itemize}
\end{definition}

\begin{definition}[Non-strict/Reflexive Partial Order]
    A non-strict partial order is a set $P$ with a relation $\leq$ such that the following properties hold.
    \begin{itemize}
        \item Antisymmetry: $a \leq b \land b \leq a \implies a = b$
        \item Transitivity: $a \leq b \land b \leq c \implies a \leq c$  
        \item Reflexivity: $a \leq a$
    \end{itemize}
\end{definition}

In the following, the definitions are given with non-strict posets, but definition for strict posets is analogous for the most part.

\begin{definition}[Hasse Diagram]
    The Hasse diagram is a directed graph with vertices $P$ where $(a, b)$ is an arc if $a \leq b$ and there is no "middle" $c$ such that $a \leq c \leq b$.
\end{definition}

\begin{definition}[Minimal/Maximal Element]
    $a$ is a min./max. element $\in P$ if 
    \[
    \forall b \in P: a \leq b \text{ (minimal element)}
    \]
    \[
    \forall b \in P: a \geq b \text{ (maximal element)}
    \]
\end{definition}

\begin{definition}[Möbius Function on Poset]
    Let $(P, \leq)$ be a locally finite poset with a single minimal element.
    $\mu: P \times P \to \mathbb{R}$ is the Möbius function of $P$ if
    \[
    \forall x,y: \sum_{z \in [x, y]} \mu(z, y) = \delta_{x, y} = 
    \begin{cases}
        0, & \text{if } x \neq y\\
        1, & \text{if } x = y
    \end{cases}
    \]
\end{definition}

\begin{definition}[Summation Function]
    Let $f: P \to \mathbb{R}$. Then the summation function $S_f(x)$ is
    \[
    S_f(x) = \sum_{z \in [0, x]} f(z)
    \]
\end{definition}

\begin{theorem}[Möbius Inversion]
    Let $f: P \to \mathbb{R}$, $S_f(x)$ is the summation function, $\mu(x, y)$ is the Möbius function of $P$. Then the following "inversion" holds.
    \[
    f(x) = \sum_{z \in [0, x]} S_f(z) \mu(z, x)
    \]
\end{theorem}

\end{document}
